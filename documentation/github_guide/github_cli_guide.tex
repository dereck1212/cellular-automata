\documentclass[11pt,a4paper]{article}
\usepackage[utf8]{inputenc}
\usepackage[T1]{fontenc}
\usepackage{lmodern}
\usepackage[french]{babel}
\usepackage{graphicx}
\usepackage{listings}
\usepackage{xcolor}
\usepackage{hyperref}
\usepackage{tcolorbox}
\usepackage{minted}
\usepackage{fancyhdr}

% Define colors
\definecolor{codegreen}{rgb}{0,0.6,0}
\definecolor{codegray}{rgb}{0.5,0.5,0.5}
\definecolor{codepurple}{rgb}{0.58,0,0.82}
\definecolor{backcolour}{rgb}{0.95,0.95,0.92}

% Configure listings
\lstdefinestyle{mystyle}{
    backgroundcolor=\color{backcolour},   
    commentstyle=\color{codegreen},
    keywordstyle=\color{magenta},
    numberstyle=\tiny\color{codegray},
    stringstyle=\color{codepurple},
    basicstyle=\ttfamily\footnotesize,
    breakatwhitespace=false,         
    breaklines=true,                 
    captionpos=b,                    
    keepspaces=true,                 
    numbers=left,                    
    numbersep=5pt,                  
    showspaces=false,                
    showstringspaces=false,
    showtabs=false,                  
    tabsize=2
}
\lstset{style=mystyle}

% Configure page style
\pagestyle{fancy}
\fancyhf{}
\rhead{Guide GitHub CLI}
\lhead{Dereck Ewane}
\rfoot{Page \thepage}

\title{Guide Complet: Gestion de GitHub via CLI}
\author{Dereck Ewane}
\date{\today}

\begin{document}

\maketitle
\tableofcontents
\newpage

\section{Introduction}
Ce guide détaillé explique comment utiliser GitHub efficacement depuis votre terminal en utilisant GitHub CLI (gh). Il couvre l'installation, la configuration, et toutes les opérations courantes que vous pouvez effectuer.

\section{Installation et Configuration}
\subsection{Installation de GitHub CLI}
Pour installer GitHub CLI sur macOS:
\begin{lstlisting}[language=bash]
brew install gh
\end{lstlisting}

\subsection{Authentification}
Pour vous connecter à votre compte GitHub:
\begin{lstlisting}[language=bash]
gh auth login
\end{lstlisting}

\subsection{Configuration SSH}
Générer une nouvelle clé SSH:
\begin{lstlisting}[language=bash]
ssh-keygen -t ed25519 -C "votre@email.com"
ssh-add ~/.ssh/id_ed25519
\end{lstlisting}

Ajouter la clé à GitHub:
\begin{lstlisting}[language=bash]
gh ssh-key add ~/.ssh/id_ed25519.pub
\end{lstlisting}

\section{Gestion des Dépôts}
\subsection{Création d'un Nouveau Dépôt}
\begin{lstlisting}[language=bash]
# Créer un nouveau dépôt
gh repo create nom-repo --public/--private

# Créer et cloner en même temps
gh repo create nom-repo --clone
\end{lstlisting}

\subsection{Cloner un Dépôt}
\begin{lstlisting}[language=bash]
gh repo clone utilisateur/nom-repo
\end{lstlisting}

\subsection{Visualiser un Dépôt}
\begin{lstlisting}[language=bash]
# Voir les informations dans le terminal
gh repo view

# Ouvrir dans le navigateur
gh repo view --web
\end{lstlisting}

\section{Gestion du Code}
\subsection{Opérations Git Basiques}
\begin{lstlisting}[language=bash]
# Ajouter des fichiers
git add .

# Créer un commit
git commit -m "message du commit"

# Pousser les modifications
git push origin main
\end{lstlisting}

\subsection{Branches}
\begin{lstlisting}[language=bash]
# Créer une nouvelle branche
git checkout -b nom-branche

# Changer de branche
git checkout nom-branche

# Fusionner une branche
git merge nom-branche
\end{lstlisting}

\section{Issues et Pull Requests}
\subsection{Gestion des Issues}
\begin{lstlisting}[language=bash]
# Créer une issue
gh issue create --title "Titre" --body "Description"

# Lister les issues
gh issue list

# Voir une issue spécifique
gh issue view numero-issue
\end{lstlisting}

\subsection{Gestion des Pull Requests}
\begin{lstlisting}[language=bash]
# Créer une PR
gh pr create --title "Titre" --body "Description"

# Lister les PRs
gh pr list

# Voir une PR spécifique
gh pr view numero-pr

# Merger une PR
gh pr merge numero-pr
\end{lstlisting}

\section{Workflows Avancés}
\subsection{GitHub Actions}
\begin{lstlisting}[language=bash]
# Voir les workflows
gh workflow list

# Exécuter un workflow
gh workflow run nom-workflow
\end{lstlisting}

\subsection{Gestion des Releases}
\begin{lstlisting}[language=bash]
# Créer une release
gh release create v1.0.0 --title "Version 1.0.0"

# Lister les releases
gh release list
\end{lstlisting}

\section{Astuces et Bonnes Pratiques}
\begin{tcolorbox}[title=Conseil 1: Commits Atomiques]
Faites des commits petits et focalisés sur une seule modification logique.
\end{tcolorbox}

\begin{tcolorbox}[title=Conseil 2: Messages de Commit]
Écrivez des messages de commit clairs et descriptifs:
\begin{itemize}
    \item Première ligne: résumé court (50 caractères max)
    \item Saut de ligne
    \item Corps: description détaillée si nécessaire
\end{itemize}
\end{tcolorbox}

\begin{tcolorbox}[title=Conseil 3: Branches]
Utilisez des branches pour chaque nouvelle fonctionnalité ou correction:
\begin{itemize}
    \item feature/nom-feature
    \item bugfix/description-bug
    \item hotfix/description-correction
\end{itemize}
\end{tcolorbox}

\section{Résolution des Problèmes Courants}
\subsection{Problèmes d'Authentification}
Si vous rencontrez des problèmes d'authentification:
\begin{lstlisting}[language=bash]
# Vérifier le statut
gh auth status

# Se reconnecter
gh auth login
\end{lstlisting}

\subsection{Conflits de Merge}
En cas de conflits:
\begin{lstlisting}[language=bash]
# Mettre à jour votre branche
git pull origin main

# Résoudre les conflits manuellement
# Puis commit les changements
git add .
git commit -m "résolution des conflits"
\end{lstlisting}

\section{Conclusion}
GitHub CLI est un outil puissant qui vous permet de gérer efficacement vos projets GitHub directement depuis votre terminal. Avec ce guide, vous avez toutes les commandes essentielles à portée de main.

\end{document}
